\documentclass[10pt]{article}

%Packages prioritaires
\usepackage[utf8]{inputenc}  \usepackage[T1]{fontenc}
\usepackage[francais]{babel} \usepackage{titlesec} \usepackage{fancyhdr} % Pour
mettre des en-têtes et des pieds de page \usepackage{array}
\usepackage{graphicx} \usepackage[margin=0.8in]{geometry} \usepackage{hyperref}
\title{\Huge{Outil de modélisation de performances des migrations de données
inter-cloud}} \author{\textbf{Antoine Martin - Carole Bonfré} } \date{Février
2015}

%Param\'{e}trage des pages
\pagestyle{fancy} \fancyhead[R]{A.Martin \& C.Bonfré} \fancyhead[L]{MIF20 - TER}
\renewcommand{\headrulewidth}{0.4pt} \renewcommand{\footrulewidth}{0.4pt}

\renewcommand{\thesection}{\Roman{section}}
\renewcommand{\thesubsection}{\Alph{subsection}}
\renewcommand{\thesubsubsection}{\arabic{subsubsection}}
\titleformat{\subsection}{\bfseries\large}{\hspace{1ex}}{1em}{\thesubsection{} \
}
\titleformat{\subsubsection}{\bfseries\normalsize}{\hspace{2ex}}{1em}{\thesubsubsection{}
\ }

\begin{document}

\maketitle

\textbf{Résumé : } Ce projet a pour but de permettre de choisir, selon une
localisation géographique et des paramètres donnés, la meilleure solution
d'hébergement "\textit{Cloud}" existante. De nombreux chercheurs rencontrent le
besoin de déplacer des Téraoctets de données, cet outil leur permettra donc
d'optimiser la migration de leurs données. A terme, ce projet Open Source
proposera une solution unique sur le marché.

\section{Introduction}

Dans un monde toujours plus interconnecté, nos données sont de plus en plus
dématérialisées et éparpillées. Nous sommes amenés à les transférer d’un
hébergeur à un autre et dans le but d’optimiser ces opérations, il nous a été
demandé de nous poser différentes questions concernant l’évaluation des méthodes
de transfert.\\

Cela s’inscrit dans le cadre de l’équipe de recherche Avalon du LIP de l’ENS de
Lyon, qui propose des solutions pour la distribution des calculs dans des
fédérations de \textit{Cloud}. La distribution de ces calculs implique de
nombreux mouvements de données inter-cloud. Pour contribuer à l’amélioration des
travaux de recherche des membres de l’équipe Avalon, nous allons développer un
outil permettant d’évaluer la performance de transferts de données entre
plusieurs hébergeurs de "\textit{Cloud}". Cet outil permettra soit d’effectuer
des tests de performances « à la demande », soit de récupérer les résultats déjà
obtenus lors de précédents tests.\\

Ce projet, nous l’espérons, pourra non seulement aider les chercheurs de
l’équipe Avalon de l’ENS mais également offrir un outil à tous projets ou
personnes ayant besoin d’évaluer les temps de transfert entre solutions
"\textit{Cloud}" (académiques comme privées).

\section{Recherche et Analyse} Une première étude des solutions existantes nous
aura permis de sélectionner les outils les plus cohérents pour définir la
structure des données de notre application.  \subsection{Etude de l'existant}

Cette étude peut être divisée en deux parties. Tout d’abords, nous avons
détaillés les offres de solution de stockage "\textit{Cloud}" disponibles sur le
marché afin de pouvoir en sélectionner. Nous avons ensuite cherchés un outil
déjà capable de réaliser des tests de performances sur des "\textit{Clouds}"
pour justifier le développement de notre propre outil.

\newpage

\subsubsection{Solutions de stockage "\textit{Clouds}"}

\begin{table}[!h] \caption{Tableau comparatif des "\textit{Clouds}"}
\renewcommand{\arraystretch}{1.5} \begin{center}
\begin{tabular}{|m{1in}|c|m{1in}|m{1in}|m{1in}|m{1in}|c|} \hline \bf\centering
drive & \bf API & \bf Emplacement & \bf Libre & \bf\centering Espace de stockage
& \bf Limitation & \bf SDK\\ \hline \centering Dropbox & Oui & S3 & Gratuit /
Propriétaire & 2Go & Oui(N/A) & Oui \\ \hline \centering Google drive & Oui  &
\href{http://www.google.com/about/datacenters/inside/locations/index.html}{lien}
& Gratuit / Propriétaire & 15Go & 10 000 requêtes /jour, 10 requêtes /sec/user &
Oui \\ \hline \centering S3 & Oui  &
\href{http://aws.amazon.com/fr/about-aws/global-infrastructure/}{lien} & Gratuit
/ Propriétaire & 5Go & 20 000 GET, 2 000 PUT / mois & Oui \\ \hline \centering
Onedrive & Oui  & ? & Gratuit / Propriétaire & 15Go & Oui (NA) & Oui \\ \hline
\centering Cloud Orange & Oui  & Paris (Sénégal ?) & Gratuit / Propriétaire &
10Go à 100Go & Oui (2Go par fichier) & Oui \\ \hline \centering Hubic & Oui  &
France (Paris, Roubaix) & Gratuit / Propriétaire & 25Go & Oui (10Go par fichier)
& Oui \\ \hline \centering Microsoft Azure & Oui  &
\href{http://azure.microsoft.com/en-us/regions/}{lien} & Gratuit / Propriétaire
& 100To & Oui (NA) & Oui \\ \hline \centering iCloud & Oui  & USA (Caroline du
Nord) & Gratuit / Propriétaire & 5Go & Oui (15Go par fichier) & Oui \\ \hline
\centering Google Cloud Storage & Oui  &
\href{http://www.google.com/about/datacenters/inside/locations/index.html}{lien}
& Payant  / Propriétaire & 1To & Oui 5Tb par fichier & Oui \\ \hline \centering
Cloud bouygues & Non  & USA (Pogoplug) & Payant  / Propriétaire & 5Gb & Oui (NA)
& Non \\ \hline \centering SFR Cloud & Non & Paris & Payant / Propriétaire &
100Go & Oui (NA) & Non \\ \hline \end{tabular} \end{center} \end{table} *La
mention Oui(N/A) pour la colonne des limitations signifie une présence de
limitation non explicitée par l'hébergeur.\\

Nous avons remarqué lors de notre analyse que l’hébergeur Dropbox utilise en
fait les services d'Amazon S3. Dropbox étant un des services les plus
populaires, nous avons décidé de le choisir tout de même car il apporte
probablement des services supplémentaires. Nous pensons qu’il est également
intéressant de tester les différences de performance entre Amazon S3 et Dropbox.
Pour les mêmes raisons, aurait également été intéressant de pouvoir vérifier les
différences entre Google Drive et Google Cloud Storage.\\

Nous avons aussi étudié la possibilité de sélectionner OwnCloud parmi les
hébergeurs (mais l’utilisateur doit posséder une machine avec OwnCloud
installé). Il ne possède pas de limitations particulières, l’espace de stockage
est “infini” (il dépend du serveur), et nous avons trouvé un SDK développé par
un tiers qui semble exploitable pour notre application. A terme, l’ajout de
OwnCloud peut donc être envisagé. Après analyse des différents acteurs du marché
de stockage en ligne, nous avons décidé de sélectionner les trois hébergeurs
suivant : Dropbox, Amazon S3 et Google Drive. Ils possèdent tous les trois des
SDK qui facilitent le développement de notre outil. En revanche, les espaces de
stockage sont parfois assez limités. Nous souhaiterions, à terme, ajouter Google
Cloud Storage pour les raisons citées précédemment.

\subsubsection{Solutions d'évaluation de performance des stockages
"\textit{Cloud}"}

Trois projets ont été identifiés durant cette étude. Le premier, HP Performance,
est une solution propriétaire payante. Parmi de nombreux outils, elle propose de
se placer dans une zone géographique pour provisionner un générateur de charges
(il est donc possible de chercher le meilleur emplacement). Le second projet,
COSBench, est une solution libre mais plutôt limitée puisqu’elle ne concerne que
Swift Storage et Amazon S3. Il permet d'exécuter des tâches sur des outils
distants et de les surveiller. Enfin, CloudScreener est une solution payante que
nous n’avons pas pu tester (nous ne connaissons donc pas la précision de cet
outil). Nous les avons comparés à notre projet baptisé KYD (Know Your Data).

\begin{table}[h] \caption{Tableau comparatif des solutions trouvées}
\renewcommand{\arraystretch}{1.5} \begin{center}
\begin{tabular}{|p{2cm}|c|p{2cm}|p{3cm}|p{2cm}|} \hline & \bf HP Performance &
\bf CosBench & \bf CloudScreener & \bf Kyd  \\ \hline \bf\centering Générique &
Oui & Oui & Non & Oui \\ \hline \bf\centering Open source & Non & Oui & Non &
Oui \\ \hline \bf\centering Modulaire & Non & Non & Probablement & Oui \\ \hline
\bf\centering Interface graphique & Oui & Oui & Oui & Non \\ \hline
\bf\centering Limites & Propriétaire & Swift Storage et S3 & Propriétaire, tests
spécifiques & limites des \textit{Clouds} \\ \hline \bf\centering Stockage des
résultats & Exports multiples & Exports multiples & Web & Base de données \\
\hline \end{tabular} \end{center} \end{table}

Un autre outil en cours d'implémentation a aussi attiré notre attention :
PerfKit. Développé par Google, il est très similaire à notre projet mais
comporte des différences majeures puisqu'il effectue ses tests seulement pour
les machines virtuelles. Il nécessite de pouvoir installer des logiciels sur les
serveurs des "\textit{Clouds}", ce qui n'est pas réalisable puisqu'il faut avoir
l'accord les hébergeurs (le projet se limite donc aux propres hébergeurs de
Google et, actuellement, à Microsoft Azure et Amazon AWS qui sont des serveurs
virtuels privés). Ce projet ne peut pas atteindre les hébergeurs que nous
ciblons et ne constitue donc pas un concurrent à proprement parler. On peut donc
remarquer que l’application que nous allons développer se situe sur un secteur
qui n’est pas encore exploité, ce qui veut dire que nous proposerons une
solution unique.

\subsection{Définition de la structure et modélisation}

Un grand nombre de paramètre est à prendre en compte pour obtenir des résultats
cohérents et réutilisables. Il faut donc connaître la taille du fichier
transféré, l'emplacement géographique de l'utilisateur, les protocoles utilisés,
la date du test, les hébergeurs à teste, ainsi que le type de transfert
("\textit{upload}" ou "\textit{download}") pour sauvegarder les résultats avec
précision. En retour, l'application propose une liste des différents hébergeurs
testés avec les temps obtenus, le meilleur choix étant mis en valeur. Tous les
résultats calculés sont stockés en base de données pour pouvoir être réutilisés
lors de tests ultérieurs.

\subsection{Choix des technologies}

Le fait de travailler avec beaucoup de paramètres différents impose de pouvoir
tester toutes les solutions possibles de manière exhaustive. Il fallait
également pouvoir stocker ces dernières de la façon la plus adapté possible pour
pouvoir les retrouver facilement. Deux outils ont donc été retenus Execo Engine
et MongoDB.

\subsubsection{Execo engine}

Execo est un outil développé par les membres de l'ENS ayant de multiples
fonctionnalités. Dans notre cas, seule la partie Engine de Execo est
intéressante puisqu'elle permet de combiner tous les paramètres passés par
l'utilisateur pour tester les différentes possibilités de test. Par exemple, si
l'utilisateur souhaite tester le transfert de fichiers de différentes tailles
sur les différents hébergeurs, Execo Engine génère les combinaisons
"taille1/drive1", "taille1/drive2", etc... Toute combinaison n'ayant pas pu être
testée à cause d'une erreur est enregistrée et peut être relancée plus tard.Il
s'agit donc d'un excellent outil de test qui chronomètre également tous les
tests effectués de façon très précise.

\subsubsection{MongoDB}

Notre application possédant une structure de données appelée à être modifiée
régulièrement, il fallait qu'elle possède un système de gestion de base de
données adapté. Sachant qu'une seule table serait nécessaire mais qu'elle
contiendrait, à terme, un très grand nombre de tuples, nous nous sommes tournés
vers le système non relationnel qu'est MongoDB. Sa vitesse de traitement
associée aux index permet de requêter très rapidement pour fournir un résultat à
l'utilisateur.

\section{Implementations}

KYD est une application implémentée en langage Python. Son développement se
divise en deux parties : l'interaction avec les hébergeurs et l'interaction avec
l'utilisateur. Pour plus d'efficacité, nous avons implémenté ensemble toutes les
parties sur Dropbox puis nous nous sommes partagés le travail sur Amazon S3 et
Google Drive.

\subsection{Interaction "\textit{Clouds}"}

Dans un premier temps, il nous a fallu mettre en place toutes les connexions
avec les hébergeurs sélectionnés. Nous avons utilisé les SDK propres à chaque
"\textit{Cloud}" et nous avons ensuite effectué une série de tests unitaires
pour vérifier le fonctionnement du transfert de fichiers en "\textit{download}"
et en "\textit{upload}".\\

Cette partie du code se veut très modulaire puisqu'elle doit permettre l'ajout
d'autres hébergeurs. Pour faciliter ces ajouts, nous avons créé une classe par
hébergeur (il suffit donc d'implémenter les fonctions désirée pour effectuer un
ajout). Nous avons rencontré quelques difficultés durant cette étape puisqu'une
telle implémentation demande de s'adapter à l'utilisation de chaque SDK. Malgré
tout, les hébergeurs sélectionnés étant très populaires, ils disposent d'une
large communauté qui permet de régler rapidement la majorité des problèmes.

\subsection{Interaction utilisateurs}

Actuellement, l'interface de communication avec l'utilisateur se fait sous
console. L'utilisateur demande à l'application la meilleure solution de
transfert de données selon ses paramètres et, si la base de données MongoDB
contient des informations suffisamment similaires, le résultat est retourné à
l'utilisateur. Dans le cas contraire, KYD demande à l'utilisateur s'il accepte
d'effectuer un test selon certains paramètres donnés pour enrichir la base de
données et pouvoir répondre lors d'une prochaine requête. Pour minimiser le
nombre de paramètres à saisir pour l'utilisateur, sa localisation géographique
est automatiquement détectée grâce à son adresse IP et à deux API de
géolocalisation (\href{http://ip-api.com/docs/api:json}{ip-api} et
\href{http://www.telize.com/}{Telize} qui se relaient si le serveur de l'une
d'entre elles est hors service).\\

Le schéma qui suit synthétise le fonctionnement de notre application dont les
divers éléments ont été expliqués précédemment.

\newpage

\begin{figure}[h] \centering \includegraphics[scale=0.5]{architecture.png}
\caption{Architecture de l'application} \label{fig:Architecture de
l'application} \end{figure}

\subsection{Améliorations}

A terme, plusieurs améliorations sont envisageables. Ajouter des paramètres pour
augmenter la précision des tests ferait partie de ces améliorations. Il serait
par exemple possible de vérifier l'efficacité des compte premium qui apportent
peut être de meilleures performances. Il faudrait également compléter la base de
données pour pouvoir répondre le plus souvent possible à l'utilisateur sans
avoir à lui faire effectuer des tests parfois coûteux en temps. Il serait alors
nécessaire d'effectuer des tests à plusieurs emplacements géographiques dans le
monde.\\

Enfin, une interface Web plus conviviale pourrait rapporter les résultats à
l'utilisateur à la place de la console. Cette amélioration se veut
majoritairement esthétique mais pourrait apporter un certain nombre de
renseignements supplémentaires à l'utilisateur.

\section{Analyse des résultats}

Suite à de nombreux tests, nous avons pu obtenir différentes courbes traduisant
les divers aspects de notre projet. Nous désirions comparer les trois hébergeurs
pour une localisation donnée, mais nous voulions aussi savoir s'il existait des
différence selon le moment de la journée ou selon le jour de la semaine.\\

Pour les figures 1 et 2, les tests ont été réalisés depuis le campus de la Doua
à Lyon 1. Un grand nombre de tests ont été effectués en "\textit{upload}" et en
"\textit{download}" avec une taille de fichier différente puis la moyenne de ces
tests a permis d'obtenir ces résultats.

\begin{figure}[h] \centering
\includegraphics[scale=0.7]{graphe_des_downloads.png} \caption{Graphe des
\textit{downloads}} \end{figure}

\newpage

\begin{figure}[h] \centering \includegraphics[scale=0.7]{graphe_des_uploads.png}
\caption{Graphe des \textit{uploads}} \end{figure}


Il est clairement visible que, dans cette situation, Dropbox est le choix le
moins pertinent sauf en ce qui concerne la connexion. Amazon et Google Drive
seront donc des choix beaucoup plus logiques puisqu'on peut observer une
différence d'environ 20 points entre Dropbox et ses deux concurrents en ce qui
concerne les performances (et cette différence ne cesse de croître avec
l'augmentation de la taille des fichiers).\\

Pour les deux figures suivantes, les tests ont été effectués depuis un serveur
virtuel situé à Londres. Ils ont été lancés toutes les heures pendant vingt
quatre heures un dimanche avec une taille de fichier de 10Mo à l'aide de la
crontab du serveur.


\begin{figure}[h] \centering
\includegraphics[scale=0.7]{graphe_du_15022015_pour_les_download_de_taille_10mo.png}
\caption{Graphe des \textit{downloads} du 15 de taille 10Mo} \end{figure}

\newpage

\begin{figure}[h] \centering
\includegraphics[scale=0.7]{graphe_du_15022015_pour_les_uploads_de_10mo.png}
\caption{Graphe des \textit{uploads} du 15 de taille 10Mo} \end{figure}


En "\textit{upload}", on observe qu'Amazon suit un schéma connu qui comporte des
pics de latence à 11h, 14h puis entre 18h et 20h. On suppose que cette situation
est due au grand nombre de connexions simultanées durant une plage horaire plus
bondée que les autres. Les performances de Dropbox sont, cette fois encore, très
nettement inférieures à celles de ses concurrents.



\section{Conclusion}

Durant ce projet, nous avons donc implémenté un outil permettant d'estimer la
meilleure solution de transfert de données entre plusieurs hébergeurs. Pour ce
faire, une phase de recherche était nécessaire puisqu'il fallait vérifier qu'un
tel outil n'existait pas déjà et, dans ce cas, sélectionner les meilleures
solutions "\textit{Cloud}".\\

Les résultats obtenus suite aux tests ont permis de vérifier la validité de
notre application. En effet, un écart trop important entre les résultats aurait
entraîné un manque d'homogénéité des moyennes. Dans une telle situation, KYD
n'aurait pas pu fournir de résultat valable et nos recherches auraient prouvées
qu'un tel outil n'était pas réalisable.\\

\section{Annexes}

\end{document}
